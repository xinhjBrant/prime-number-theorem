%% Generated by Sphinx.
\def\sphinxdocclass{report}
\documentclass[letterpaper,10pt,english]{sphinxmanual}
\ifdefined\pdfpxdimen
   \let\sphinxpxdimen\pdfpxdimen\else\newdimen\sphinxpxdimen
\fi \sphinxpxdimen=.75bp\relax
\ifdefined\pdfimageresolution
    \pdfimageresolution= \numexpr \dimexpr1in\relax/\sphinxpxdimen\relax
\fi
%% let collapsible pdf bookmarks panel have high depth per default
\PassOptionsToPackage{bookmarksdepth=5}{hyperref}

\PassOptionsToPackage{warn}{textcomp}
\usepackage[utf8]{inputenc}
\ifdefined\DeclareUnicodeCharacter
% support both utf8 and utf8x syntaxes
  \ifdefined\DeclareUnicodeCharacterAsOptional
    \def\sphinxDUC#1{\DeclareUnicodeCharacter{"#1}}
  \else
    \let\sphinxDUC\DeclareUnicodeCharacter
  \fi
  \sphinxDUC{00A0}{\nobreakspace}
  \sphinxDUC{2500}{\sphinxunichar{2500}}
  \sphinxDUC{2502}{\sphinxunichar{2502}}
  \sphinxDUC{2514}{\sphinxunichar{2514}}
  \sphinxDUC{251C}{\sphinxunichar{251C}}
  \sphinxDUC{2572}{\textbackslash}
\fi
\usepackage{cmap}
\usepackage[T1]{fontenc}
\usepackage{amsmath,amssymb,amstext}
\usepackage{babel}



\usepackage{tgtermes}
\usepackage{tgheros}
\renewcommand{\ttdefault}{txtt}



\usepackage[Bjarne]{fncychap}
\usepackage{sphinx}

\fvset{fontsize=auto}
\usepackage{geometry}


% Include hyperref last.
\usepackage{hyperref}
% Fix anchor placement for figures with captions.
\usepackage{hypcap}% it must be loaded after hyperref.
% Set up styles of URL: it should be placed after hyperref.
\urlstyle{same}

\addto\captionsenglish{\renewcommand{\contentsname}{Contents:}}

\usepackage{sphinxmessages}
\setcounter{tocdepth}{1}



\title{Formal Proof of Prime Number Theory in Lean}
\date{Aug 05, 2022}
\release{v0}
\author{}
\newcommand{\sphinxlogo}{\vbox{}}
\renewcommand{\releasename}{Release}
\makeindex
\begin{document}

\pagestyle{empty}
\sphinxmaketitle
\pagestyle{plain}
\sphinxtableofcontents
\pagestyle{normal}
\phantomsection\label{\detokenize{index::doc}}



\chapter{Introduction}
\label{\detokenize{Blueprint/Introduction:introduction}}\label{\detokenize{Blueprint/Introduction::doc}}

\chapter{Definitions}
\label{\detokenize{Blueprint/Definitions:definitions}}\label{\detokenize{Blueprint/Definitions::doc}}

\chapter{Contour Integral}
\label{\detokenize{Blueprint/Contour_Integral:contour-integral}}\label{\detokenize{Blueprint/Contour_Integral::doc}}

\section{I. Definitions of Paths}
\label{\detokenize{Blueprint/Contour_Integral:i-definitions-of-paths}}
\sphinxAtStartPar
A path is a differentiable function \(f : \mathbb{R}\to \mathbb{C}\) with a continuous derivative, defined on \(\mathbb{R}\), but we only consider their values on {[}0,1{]}.
\begin{description}
\item[{Constant Paths}] \leavevmode
\sphinxAtStartPar
For example, for any fixed point \(z : \mathbb{C}\) we have a constant path \(\lambda (t : \mathbb{R}), z\).

\item[{Inverse of Paths}] \leavevmode
\sphinxAtStartPar
The inverse of a path is to reverse the direction of a path.

\sphinxAtStartPar
For example, given a path \(L: \mathbb{R}\to \mathbb{C}\), we define the inverse of the path as \(\lambda (t:\mathbb{R}), L(1-t)\).

\end{description}


\chapter{Indices and tables}
\label{\detokenize{index:indices-and-tables}}\begin{itemize}
\item {} 
\sphinxAtStartPar
pdf download : \sphinxhref{../latex/formalproofofprimenumbertheoryinlean.pdf}{pdf}

\item {} 
\sphinxAtStartPar
\DUrole{xref,std,std-ref}{genindex}

\item {} 
\sphinxAtStartPar
\DUrole{xref,std,std-ref}{search}

\end{itemize}



\renewcommand{\indexname}{Index}
\printindex
\end{document}